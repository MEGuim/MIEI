%
% Layout retirado de http://www.di.uminho.pt/~prh/curplc09.html#notas
%
\documentclass{report}
\usepackage[portuges]{babel}
\usepackage[utf8]{inputenc}

\usepackage{url}
\usepackage{alltt}
\usepackage{fancyvrb}
\usepackage{listings}
%LISTING - GENERAL
\lstset{
	basicstyle=\small,
	numbers=left,
	numberstyle=\tiny,
	numbersep=5pt,
	breaklines=true,
    frame=tB,
	mathescape=true,
	escapeinside={(*@}{@*)}
}

\usepackage{xspace}

\parindent=0pt
\parskip=2pt

\setlength{\oddsidemargin}{-1cm}
\setlength{\textwidth}{18cm}
\setlength{\headsep}{-1cm}
\setlength{\textheight}{23cm}

\def\darius{\textsf{Darius}\xspace}
\def\java{\texttt{Java}\xspace}
\def\pe{\emph{Publicação Eletrónica}\xspace}

\begin{document}

\title{Universidade do Minho\\Escola de Engenharia\\Departamento de Informática\\Processamento de Linguagens - 3º ano MIEI\\ \rule{15cm}{1pt}\\ \textbf{Trabalho Prático 2} \\ Compilador para Linguagem Imperativa Simples \rule{15cm}{1pt}}

\author{Gonçalo Pinheiro\\ (66548) \and Miguel Guimarães\\ 66822 \and Nuno Fernandes\\ 61066}

\date{\today}

\maketitle

\begin{abstract}
Este relatório tem como objectivo demonstrar todo o processo derivado da construção de um compilador para uma linguagem de programação imperativa simples (lpis). O compilador irá, através da análise sintática e semântica de um ficheiro escrito na lpis criada, gerar código fonte para a maquina virtual disponibilizada pelos docentes.
\end{abstract}

\tableofcontents

\newpage

\chapter{Introdução} \label{intro}
Este documento aborda o processo de construção de um compilador de uma linguagem de programação imperativa simples que é, também, detalhada neste relatório.

\section{Contextualização do Problema} \label{ctp}
Por forma a consolidar os conhecimentos adquiridos na Unidade Curricular de Processamentos de Linguagens foi proposto a criação de um compilador de uma linguagem de programação imperativa juntamente com a sua definição. \\ Em primeiro lugar, é necessário determinar os passos a efetuar e quais as ferramentas que serão utilizadas neste processo. Após este passo, trata-se de definir a linguagem pretendida. \\ Em seguida, é necessário criar ferramentas que irão ler os ficheiros de código fonte e validar todo o código que é introduzido na linguagem anteriormente definida para a traduzir (ou não, caso seja inválida) para pseudo-código Assembly que é interpretado pela Máquina Virtual disponibilizada pelos docentes. Este tratamento é inicializado pela ferramenta flex que tem como objectivo fazer a análise léxica do código e devolver os símbolos que encontra. Termina, então, quando se faz a análise sintática e semântica, utilizando a ferramenta YACC, que verifica se o código fonte respeita a linguagem e se segue as regras semânticas definidas. \\ Com esta estrutura, espera-se que o resultado final de cada código fonte introduzido seja o pseudo-código com as funcionalidades esperáveis.

\section{Objetivos}
A elaboração deste projeto tem objetivos que foram claramente enunciados que são: 
\begin{itemize}
	\item aumentar a experiência em engenharia de linguagens e em programação generativa (gramatical);
	\item desenvolver processadores de linguagens segundo o método da tradução dirigida pela sintaxe, suportado numa gramática tradutora;
	\item desenvolver um \textbf{compilador} gerando código para uma \textbf{máquina de stack virtual};
	\item utilizar geradores de compiladores baseados em gramáticas tradutoras, concretamente o YACC;
	\item aumentar a experiência de uso do ambiente Linux, da linguagem imperativa C (para codificação das estruturas
	de dados e respectivos algoritmos de manipulação), e de algumas ferramentas de apoio à programação;
	\item rever e aumentar a capacidade de escrever gramáticas independentes de contexto que satisfaçam a condição LR()
	usando BNF-puro.
\end{itemize}
Após a definição e conhecimento dos objetivos que são propostos, é necessário passar à produção da nossa proposta de solução.


\chapter{LPIS}

\section{Requisitos}
Tal como nos objetivos, os requisitos foram facultados no enunciado e são os seguintes:
\begin{itemize}
	\item declarar e manusear variáveis atómicas do tipo inteiro, com os quais se podem realizar as habituais operações aritméticas, relacionais e lógicas;
	\item declarar e manusear variáveis estruturadas do tipo array (a 1 ou 2 dimensões) de inteiros, aos quais é apenas permitida a operação de indexação (índice inteiro).
	\item efetuar instruções algorítmicas básicas como a atribuição de expressões a variáveis.
	\item ler do standard input e escrever no standard output.
	\item efectuar instruções para controlo do fluxo de execução - condicional e cíclica - que possam ser aninhadas.
	\item definir e invocar subprogramas sem parâmetros mas que possam retornar um resultado atómico (opcional).
\end{itemize}

\section{Definição da Linguagem}
\subsection{Definição Textual da Linguagem}
A linguagem terá de seguir, com a ordenadamente, a estrutura seguinte:

\begin{itemize}
	\item É necessário iniciar com a palavra BEGINCMM e terminar com ENDCMM;
	\item Caso sejam necessárias variáveis, introduz-se a sua definição dentro de um bloco que começa por VARBEGIN e termina por VAREND
	\item O bloco de instruções é inicializado por FUNBEGIN e terminado com FUNEND
	\item As instruções e as declarações são terminadas por ;
\end{itemize} 

Após a determinação da estrutura geral de um ficheiro de código fonte da linguagem, é necessário definir os componentes interiores como as instruções e expressões que são permitidas. \\ Como tal, decidiu-se que seriam permitidas as seguintes:
\begin{itemize}
	\item Atribuição de uma expressão a uma variável;
	\item Operação de ciclo While;
	\item Operação de ciclo For;
	\item Operação condicional If;
	\item Leitura do teclado;
	\item Escrita no ecrã.
\end{itemize}
Para cada uma foi definida uma estrutura própria que cada código fonte criado deve seguir por forma a gerar o pseudo-código Assembly correspondente corretamente. \\De seguida é apresentada a estrutura para todos os tipos que se decidiram implementar:
\begin{itemize}
	\item \textbf{Atribuição:} Uma instrução de atribuição deve ser iniciada pela variável à qual se quer fazer a atribuição, seguida pelo carater '=' e terminada pela expressão que fornece o valor;
	\item \textbf{While:} O ciclo While é definido similarmente à sua declaração na linguagem de programação \textit{C}, ou seja, inicia-se a sua declaração com a palavra WHILE, seguida pelo seu bloco de condições e finalmente as instruções. O bloco de condições deve ser limitado pelos caráteres '(' e ')' para iniciar e terminar, respectivamente. O bloco de instruções deve ser limitado por '{' e '}';
	\item \textbf{For:} O ciclo For também teve como inspiração o seu homônimo de \textit{C}. Ou seja, o ciclo é iniciado pela palavra FOR seguida por uma atribuição, uma instrução e a atribuição final. Cada um dos elementos anteriores, é delimitado pelos caráteres '(' ')'. Após estas definições é inserido o bloco de instruções, iniciado por '{' e terminado por '}';
	\item \textbf{IF:} Esta instrução é iniciada pela palavra IF, o carater '(', uma condição, o carater ')' e com '{', um conjunto de instruções e '}'. Pode ser completada com um else, que significa a alternativa, iniciada com ELSE e mais uma vez delimitadas as instruções pelos carateres '{' e '}';
	\item \textbf{Leitura} A instrução de leitura é iniciada pela palavra READ seguida pela variável para onde se quer fazer a leitura delimitada por '(' e ')';
	\item \textbf{Escrita:} A instrução de escrita inicia-se pela palavra WRITE seguida de uma expressão delimitada por '(' e ')'. No caso de se querer a escrita de uma linha de texto, substitui-se no local da expressão pelo texto pretendido delimitado pelo carater '"'.
\end{itemize}
Posto isto, falta apenas descrever o que são as condições e as expressões desta linguagem. \\As condições são relações ente expressões utilizando operadores lógicos ou relacionais.\\As expressões tratam-se das simplesmente de operações aritméticas na matemática.\\De referir que em ambos os casos, nas condições e nas expressões, existe tratamento de prioridades próprios deste tipo de operações.








\subsection{Gramática Independente de Contexto}

\textbf{Terminais = \{}
\begin{verbatim}
    '{', '}', NUM, ';', '=', '+', '-', '*', '/', '(', ')', '[', ']',
    variavel, '<', '>', '>=', '<=', '==', '!=', "OR", "AND"
\end{verbatim}{} 


\textbf{\}} \\\\
\textbf{Não Terminais = \{} 
\begin{verbatim}
Codigo, Funcao, BlocoDeclaracoes, Declaracoes, Declaracao, Execucoes, Execucao,
Atribuicao, Var, Operacao, Termo, Fator, Ciclo, If, Else, BlocoCond, ExpLogica1, 
ExpLogica2, ExpLogica3, OperacaoLogica, StdInOut
\end{verbatim}
\textbf{\}}\\\\
\textbf{Axioma = }Codigo \\\\
\newpage
\textbf{Produções = \{}
\begin{verbatim}

Codigo 	: BEGINCMM  Declaracoes Funcao ENDCMM
	   	;

Funcao 	: FUNBEGIN  BlocoDeclaracoes Execucoes FUNEND
	   	|
	   	;

BlocoDeclaracoes 	: VARBEGIN Declaracoes VAREND
					;

Declaracoes : Declaracoes Declaracao
			| Declaracao
			;

Declaracao 	: variavel ';'							
			| variavel '[' NUM ']' ';'				
			| variavel '[' NUM ']' '[' NUM ']' ';'	
			;


Execucoes 	: Execucoes Execucao
			| Execucao
			;

Execucao  	: Atribuicao ';'
		  	| Ciclo
		  	| If
		  	| StdInOut ';'
		  	| ChamaFuncao ';'
		  	|
		  	;

Atribuicao  : Var '=' Operacao
			;

Var 	: variavel 											
		| variavel 	'[' Operacao ']'						
		| variavel  '[' Operacao ']' '[' Operacao ']'		
		;


Operacao  	: Termo
		  	| Operacao '+' Termo				
		  	| Operacao '-' Termo				
		  	;

Termo 	: Fator
	  	| Termo '*' Fator						
	 	| Termo '/' Fator						
	 	;

Fator 	: Var 									
	  	| NUM									
	  	| '(' Operacao ')'
	  	| READ '(' ')'							
	  	;

Ciclo 	: WHILE  '(' BlocoCond ')' '{' Execucoes '}'
	  	| FOR '(' Atribuicao ')' '(' BlocoCond ')'  '(' Operacao ')' '{' Execucoes '}'
	  	;


If 	: IF '(' BlocoCond ')' '{' Execucoes '}' Else
   	;

Else :  ELSE '{' Execucoes '}'
	 |			  			  
	 ;



BlocoCond	: BlocoCond OR ExpLogica1			
			| ExpLogica1
			;

ExpLogica1	: ExpLogica1 AND ExpLogica2			
			| ExpLogica2
			;

ExpLogica2	: NOT '(' BlocoCond ')'		
			| ExpLogica3
			;

ExpLogica3 	: OperacaoLogica
			| '(' BlocoCond ')'
			;

OperacaoLogica: Operacao '<' Operacao		
			  | Operacao '=' '<' Operacao	
			  | Operacao '<' '=' Operacao	
			  | Operacao '>' Operacao		
			  | Operacao '>' '=' Operacao	
			  | Operacao '=' '>' Operacao	
			  | Operacao '=' '=' Operacao	
			  | Operacao '!' '=' Operacao	
			  ;

StdInOut 	: WRITE '(' Operacao ')'				
		 	| WRITE '(' string ')'					
		 	;
\end{verbatim}
\textbf{\}}\\\\


\subsection{Programas de Exemplo}
De forma a melhor entender a nossa linguagem escrevemos uns programas exemplo, para mostrar as funcionalidades implementadas.
\textbf{Exemplo 1 }: Imprimir números de 1 até 10
\begin{verbatim}
BEGINCMM
	VARBEGIN
		i;
		numeros[10];
	VAREND
	
	FUNBEGIN
		FOR(i = 1) (i <= 10) (i++) {
			numeros[i-1]=i;
			WRITE(i)
		}
	FUNEND

ENDCMM

\end{verbatim}

\textbf{Exemplo 2 }: Imprimir texto, ler do standard input e demonstração de uso de condições lógicas:
\begin{verbatim}
BEGINCMM
	VARBEGIN
		i;
		aux;
		numeros[10];
	VAREND
	
	FUNBEGIN
		FOR(i = 0) (i < 10) (i++) {
			numeros[i]=READ();
		}
		i = 0;
		WHILE (i < 10 AND aux == 0) {
			IF (numeros[i] == 5) {
				aux = 1;
			} ELSE {
				i = i + 1;
			}
		}
		IF (aux != 5) {
			WRITE("Numero 5 nao encontrado");
		} ELSE {
			WRITE("5 encontrado");
		}
	FUNEND

ENDCMM
\end{verbatim}

\textbf{Exemplo 3 }: Inicializar e preencher array bidimensional:
\begin{verbatim}
BEGINCMM
	VARBEGIN
		i;
		j;
		numeros[10][10];
	VAREND
	
	FUNBEGIN
		FOR(i = 0) (i < 10) (i++) {
			FOR (j = 0) (j < 10) (j++) {
				numeros[i][j]= j + i * 10;
			}
		}
	FUNEND

ENDCMM
\end{verbatim}




\subsection{Formalização da Definição da Linguagem}
Nesta parte do documento serão identificadas as expressões regulares necessárias para a análise léxica do texto fonte. Para isto são analisados os símbolos terminais da gramática anteriormente definida e, concetualizadas as expressões regulares que permitem identificar esses símbolos. 
Assim sendo, de seguida, apresentam-se os símbolos terminais da gramática anteriormente identificados com a respetiva expressão regular para os intercetar.
\begin{verbatim}
'%' : %
'{' : {
'}' : }
NUM : [0-9]+
';' : ;
'=' : =
'+' : +
'-' : -
'*' : *
'/' : /
'(' : (
')' : )
'[' : [
']' : ]
variavel : [a-zA-Z][a-zA-Z0-9]*
'<' : < 
'>' : >
'>=' : ">="
'<=' : "<="
'==' : "=="
'!=' : "!="
'AND' : "AND"
'||' : "||"
\end{verbatim}

As palavras reservadas da linguagem são identificadas pela própria palavra, o que significa que estas palavras terão sempre um significado próprio e não poderão ser atribuídas a variáveis.
Assim sendo foi especificado o seguinte analisador léxico:

\begin{verbatim}
#include "y.tab.h"
%}
letra       [a-zA-Z]
digito      [0-9]
indesejado  \t|\n

%option yylineno
%%

[\{\}\(\)\[\]\=\<\>\,\;\"\+\-\*\/\%\&\!\|]      {return yytext[0];}


"BEGINCMM"				                        {return BEGINCMM;}
"ENDCMM"				                        {return ENDCMM;}
"VARBEGIN"				                        {return VARBEGIN;}
"VAREND"				                        {return VAREND;}
"FUNBEGIN"				                        {return FUNBEGIN;}
"FUNEND"				                        {return FUNEND;}
"FUNC"					                        {return FUNC;}

"IF"					                        {return IF;}
"ELSE"					                        {return ELSE;}
"WHILE"					                        {return WHILE;}
"FOR"					                        {return FOR;}
"AND"											{return AND;}
"OR"											{return OR;}

"READ"					                        {return READ;}
"WRITE"					                        {return WRITE;}


{letra}({letra}|{digito})*                      {yylval.pal = strdup(yytext); return variavel;}
-?{digito}+ 				                    {yylval.val = atoi(yytext); return NUM;}
\"[^\"]+\"										{yylval.pal = strdup(yytext); return string;}

{indesejado}                                    {;}

%%

int yywrap() {
   return(1);
}
\end{verbatim}

\section{Conclusão}
O objectivo deste trabalho não foi totalmente cumprido uma vez que não foi possivel implementar a funcionalidade que permite chamar funções  principalmente devido a uma pobre gestão do tempo.
Com este trabalho foi possivel consolidar a matéria dada durante o semestre de processamento de linguagens uma vez que foi necessário utilizar conhecimento adquirido em todas as aulas sem excepção.
Apesar das várias dificuldades sentidas, nomeadamente a construção da gramática tradutora e o armazenamento de vectores quer unidimensionais quer bidimensionais, consegui-se ainda assim tirar uma avaliação bastante positiva do trabalho.




\appendix
\section{Analisador Léxico .l}

\begin{lstlisting}
%{


#include "y.tab.h"
%}
letra       [a-zA-Z]
digito      [0-9]
indesejado  \t|\n

%option yylineno
%%

[\{\}\(\)\[\]\=\<\>\,\;\"\+\-\*\/\%\&\!\|]      {return yytext[0];}

"BEGINCMM"				                        {return BEGINCMM;}
"ENDCMM"				                        {return ENDCMM;}
"VARBEGIN"				                        {return VARBEGIN;}
"VAREND"				                        {return VAREND;}
"FUNBEGIN"				                        {return FUNBEGIN;}
"FUNEND"				                        {return FUNEND;}
"FUNC"					                        {return FUNC;}

"IF"					                        {return IF;}
"ELSE"					                        {return ELSE;}
"WHILE"					                        {return WHILE;}
"FOR"					                        {return FOR;}
"AND"											{return AND;}
"OR"											{return OR;}

"READ"					                        {return READ;}
"WRITE"					                        {return WRITE;}


{letra}({letra}|{digito})*                      {yylval.pal = strdup(yytext); return variavel;}
-?{digito}+ 				                    {yylval.val = atoi(yytext); return NUM;}
\"[^\"]+\"										{yylval.pal = strdup(yytext); return string;}

{indesejado}                                    {;}

%%

int yywrap() {
   return(1);
}
\end{lstlisting}

\newpage
\section{Analisador Sintático, Semântico e Gerador de Código .y}
\scriptsize\begin{verbatim}
%{
#define _GNU_SOURCE
#include <stdio.h>
#include <stdlib.h>
#include <stdarg.h>
#include "varHash.h"
#include "stack.h"

extern int yylex();
extern int yylineno;
int yyerror(char* s);

#define MAX_SIZE 1000

/*Tipos de dados*/
#define INT 3000
#define INT_ARRAY 3001

void declaracao(char* name,int size1,int size2);
int atribuicao(char* nome);
int atribuicaoArray(char* nome, int size1);
int atribuicaoAA(char* nome, int size1, int size2);
void addStartWhileLabel();
void addLoopWhileLabel();
void addEndWhileLabel();
void addStartForLabel();
void addLoopForLabel();
void addEndForLabel();
void addStartIfLabel();
void addMidIfLabel();
void addJumpEndIfLabel();
void addElseLabel();
void addEndIfElseLabel();

varHash hash;

FILE *f;

PtStk stackWhileLoop;
PtStk stackForLoop;
PtStk stackIfcondition;

int whilelabel = 1;
int forlabel = 1;
int ifcondition = 1;
int currPointer;
char *currFunc;

%}

%token variavel string NUM
%token BEGINCMM ENDCMM VARBEGIN VAREND FUNBEGIN FUNEND FUNC 
%token IF ELSE WHILE FOR
%token AND OR NOT 
%token READ WRITE


%type <val> NUM
%type <val> Var
%type <val> Operacao
%type <val> Termo
%type <val> Fator

%type <pal> variavel
%type <pal> string





%union{
	int val;
	char* pal;
}

%start Codigo

%%


Codigo 	: BEGINCMM BlocoDeclaracoes { fprintf(f, "start\n"); } Funcao ENDCMM
	   	;

Funcao 	: FUNBEGIN {/*fprintf(f,"pusha main\ncall\n");*/} BlocoDeclaracoes Execucoes FUNEND
	   	;

BlocoDeclaracoes 	: 
					| VARBEGIN Declaracoes VAREND
					;

Declaracoes : Declaracoes Declaracao
			| Declaracao
			;

Declaracao 	: variavel ';'									{ declaracao($1, 0, 0); }
			| variavel '[' NUM ']' ';'						{ if($3 < 1) printf("Erro na definição de Array"); declaracao($1, $3, 0); }
			| variavel '[' NUM ']' '[' NUM ']' ';'			{ if($3 < 1 || $6 < 1) printf("Erro na definição do Array");declaracao($1, $3, $6); }
			|
			;


Execucoes 	: Execucoes Execucao
			| Execucao
			;

Execucao  	: Atribuicao ';'
		  	| Ciclo
		  	| If
		  	| StdInOut ';'
		  	|
		  	;

Atribuicao  : Var '=' Operacao 								{ fprintf(f,"storeg %d\n", $1);}					
			;

Var 	: variavel 											{ $$ = atribuicao($1); }
		| variavel 	'[' Operacao ']'						{ $$ = atribuicaoArray($1,$3); }
		| variavel  '[' Operacao ']' '[' Operacao ']'		{ $$ = atribuicaoAA($1,$3,$6); }
		;


Operacao  	: Termo
		  	| Operacao '+' Termo				{fprintf(f,"add\n");}
		  	| Operacao '-' Termo				{fprintf(f,"sub\n");}
		  	;

Termo 	: Fator
	  	| Termo '*' Fator						{fprintf(f,"mul\n");}
	 	| Termo '/' Fator						{fprintf(f,"div\n");}
	 	;

Fator 	: Var 									{ fprintf(f,"loadn\n");}
	  	| NUM									{ fprintf(f,"pushi %d\n", $1);}
	  	| '(' Operacao ')'						{}
	  	| READ '(' ')'							{ fprintf(f,"read\natoi\nstoren\n");}
	  	;

Ciclo 	: WHILE {addStartWhileLabel();} '(' BlocoCond ')'{addLoopWhileLabel();} '{' Execucoes '}' {addEndWhileLabel();}
	  	| FOR {addStartForLabel();}'(' Atribuicao ')' '(' BlocoCond ')' {addLoopForLabel();} '(' Atribuicao ')' '{' Execucoes '}' {addEndForLabel();}
	  	;

If 	: IF {addStartIfLabel();} '(' BlocoCond ')' {addMidIfLabel();} '{' Execucoes '}' {addJumpEndIfLabel();addElseLabel();}  Else
   	;

Else :  ELSE '{' Execucoes '}' {addEndIfElseLabel();}
	 |			  			   {addEndIfElseLabel();}
	 ;



BlocoCond	: BlocoCond OR ExpLogica1			{fprintf(f, "add\npushi 0\nsup\n");}
			| ExpLogica1
			;

ExpLogica1	: ExpLogica1 AND ExpLogica2			{fprintf(f, "add\npushi 2\nequal\n");}
			| ExpLogica2
			;

ExpLogica2	: NOT '(' BlocoCond ')'			{fprintf(f, "not\n");}
			| ExpLogica3
			;

ExpLogica3 	: OperacaoLogica
			| '(' BlocoCond ')'
			;

OperacaoLogica: Operacao '<' Operacao		{fprintf(f, "inf\n");}
			  | Operacao '=' '<' Operacao	{fprintf(f, "infeq\n");}
			  | Operacao '<' '=' Operacao	{fprintf(f, "infeq\n");}
			  | Operacao '>' Operacao		{fprintf(f, "sup\n");}
			  | Operacao '>' '=' Operacao	{fprintf(f, "supeq\n");}
			  | Operacao '=' '>' Operacao	{fprintf(f, "supeq\n");}
			  | Operacao '=' '=' Operacao	{fprintf(f, "equal\n");}
			  | Operacao '!' '=' Operacao	{fprintf(f, "equal\nnot\n");}
			  ;


StdInOut 	: WRITE '(' Operacao ')'				{ fprintf(f,"writei\n");}
		 	| WRITE '(' string ')'					{ fprintf(f,"pushs %s\nwrites\n", $3);}
		 	;

%%

void declaracao(char *nome,int size1,int size2){
	int tipo,i;
	char *temp;
	if(size1==0 && size2==0) tipo=INT;	else tipo=INT_ARRAY;

	if(existeVar(&hash, nome, currFunc) != -1 || existeVar(&hash, nome, "main") != -1){
		fprintf(stderr,"ERRO linha %d: Variavel %s já definida\n",yylineno,nome);
		exit(0);
	}
   else{
       if(tipo==INT){
        	fprintf(f,"pushi 0\n");
			adVar(&hash,nome,tipo,0,currPointer,"main");
       		currPointer++;
	    }else{
			if(size2==0){
				fprintf(f,"pushn %d\n", size1);
				adVar(&hash,nome,tipo,size1,currPointer,"main");
				currPointer+=size1;
			}else{
				adVar(&hash,nome,INT,0,currPointer,"main");
				for(i=0;i<size2;i++){
					fprintf(f,"pushn %d\n", size1);
					asprintf(&temp,"%s[%d]",nome,i);
					adVar(&hash,temp,tipo,size1,currPointer,"main");
					currPointer+=size1;
				}
			}
		}
    }
}



int atribuicao(char* nome){
	int end;

	if (existeVar(&hash,nome,"main") == -1) {
		fprintf(stderr,"ERRO linha %d: Variavel %s não definida\n",yylineno,nome);
		exit(0);
	}

	varInHash v = (varInHash)malloc(sizeof(varInHash));
	v = tiraVar(&hash,nome,"main");
	end = v->endereco;

	if(strcmp(currFunc, "main"))
		fprintf(f,"pushfp\npushi %d\nadd\n", end);
	else 
		fprintf(f,"pushgp\npushi %d\nadd\n", end);

	return end;
}

int atribuicaoArray(char *nome, int size1){
	int end;

	if (existeVar(&hash,nome,"main") == -1) {
		fprintf(stderr,"ERRO linha %d: Variavel %s não definida\n",yylineno,nome);
		exit(0);
	}

	varInHash v = (varInHash)malloc(sizeof(varInHash));
	v = tiraVar(&hash,nome,"main");
	end = v->endereco+size1;

	if(strcmp(currFunc,"main"))
		fprintf(f,"pushfp\npushi %d\nadd\n", end);
	else 
		fprintf(f,"pushgp\npushi %d\nadd\n", end);

	return end;
}

int atribuicaoAA(char *nome, int size1, int size2){
	char* temp;
	int end;
	asprintf(&temp,"%s[%d]",nome,size1);

	if (existeVar(&hash,temp,"main") == -1) {
		fprintf(stderr,"ERRO linha %d: Variavel %s não definida\n",yylineno,nome);
		exit(0);
	}

	varInHash v = (varInHash)malloc(sizeof(varInHash));
	v = tiraVar(&hash,temp,"main");
	end = v->endereco+size2;
	if(strcmp(currFunc,"main"))
		fprintf(f,"pushfp\npushi %d\nadd\n", end);
	else
		fprintf(f,"pushgp\npushi %d\nadd\n", end);
	
	return end;
}

void addStartWhileLabel(){
	//fprintf(f, "\\\\ While loop \n");
	fprintf(f, "while_loop%d: NOP \n",whilelabel);
	stackPush(stackWhileLoop,whilelabel);
	whilelabel++;
}

void addLoopWhileLabel(){
	fprintf(f, "JZ end_while_loop%d \n", stackTop(stackWhileLoop));
}

void addEndWhileLabel(){
	fprintf(f, "JUMP while_loop%d \n", stackTop(stackWhileLoop));
	fprintf(f, "end_while_loop%d: NOP \n", stackTop(stackWhileLoop));
	stackPop(stackWhileLoop);
}

void addStartForLabel(){
	//fprintf(f, "\n\\\\ For Loop \n");
	fprintf(f, "for_loop%d: NOP \n", forlabel);
	stackPush(stackForLoop,forlabel);
	forlabel++;
}

void addLoopForLabel(){
	fprintf(f, "JZ end_for_loop%d \n", stackTop(stackForLoop));
}

void addEndForLabel(){
	fprintf(f, "JUMP for_loop%d \n",stackTop(stackForLoop));
	fprintf(f, "end_for_loop:%d",stackTop(stackForLoop));
	stackPop(stackForLoop);
}

void addStartIfLabel(){
	//fprintf(f, "\\\\ If Condition \n");
	fprintf(f, "ifcondition%d: NOP \n", ifcondition);
	stackPush(stackIfcondition,ifcondition);
	ifcondition++;
}

void addMidIfLabel(){
	fprintf(f, "JZ elsestatement%d\n",stackTop(stackIfcondition));
}

void addJumpEndIfLabel(){
	fprintf(f, "JUMP endif%d\n", stackTop(stackIfcondition));
}

void addElseLabel(){
	fprintf(f, "elsestatement%d: NOP \n",stackTop(stackIfcondition));
}

void addEndIfElseLabel(){
	fprintf(f, "endif%d: NOP \n", stackTop(stackIfcondition));
	stackPop(stackIfcondition);
}

int yyerror(char *s)
{
   fprintf(stderr, "ERRO: %s, linha: %d\n", s, yylineno);
   return 0;
}
int main(int argc, char *argv[]) {

	if(argc != 3){
		fprintf(stderr,"ERRO linha %d: Erro de nº de argumentos\n",yylineno);
			exit(0);
	}
	stdin = fopen(argv[1], "r");

	f = fopen(strdup(argv[2]), "w");

	currFunc = strdup("main");
	currPointer = 0;


	hash = initHash();

	stackWhileLoop = stackCreate();
	stackForLoop = stackCreate();
	stackIfcondition = stackCreate();

	yyparse();
	fprintf(f, "stop\n");
	fclose(f);

	stackDestroy(stackWhileLoop);
	stackDestroy(stackForLoop);
	stackDestroy(stackIfcondition);

	return 0;
}

\end{verbatim}
\newpage



\section{Tabela de Variaves}
\subsection{.h}
\begin{lstlisting}
#ifndef VARHASH
#define VARHASH

#include <string.h>
#include <stdlib.h>
#include <stdio.h>
#include <ctype.h>


#define MIN_SIZE 10
/*Tipos de dados*/
#define INT 3000
#define INT_ARRAY 3001


typedef struct tabelaHash *varHash;

typedef struct variable 
{
	char *nome;
	int tipo;
	int size;
	int endereco;
	char *funcao;
}*varInHash;



varHash initHash();

void removeVarHash(varHash v);

int existeVar(varHash *h, char* nome, char* funcao);

int adVar(varHash *h, char* nome, int tipo, int size, int endereco, char* funcao);

varInHash tiraVar(varHash *h, char* nome, char *funcao);

void printHash(varHash *h);

#endif
\end{lstlisting}

\subsection{.c}
\begin{lstlisting}
#include "varHash.h"

typedef struct tabelaHash
{
	int size;
	int n_elems;
	varInHash *variaveis;
}*varHash;

int getPos(int size, char* str) {

    unsigned int hash = 5381;
    int c;

    while ( (c = *str++) )
        hash = ((hash << 5) + hash) ^ c;

    return hash % size;
}

varHash initHash() {

	varHash h 	 = (varHash)malloc(sizeof(struct tabelaHash));
	h->size 	 = MIN_SIZE;
	h->n_elems 	 = 0;
	h->variaveis = (varInHash *)malloc(sizeof(struct variable) * MIN_SIZE);

	for (int i = 0; i < MIN_SIZE; i++)
	{
		h->variaveis[i] = NULL;
	}

	return h;
}

void resize_varHash(varHash *h){

	int new_size  = (*h)->size * 2;
	int pos;
	varHash novo  = initHash();
	novo->n_elems = (*h)->n_elems;
	novo->size    = new_size;

	for (int i = 0; i < (*h)->size; i++)
	{
		varInHash v = (*h)->variaveis[i];
		if (v)
		{
			pos = existeVar(&novo, v->nome, v->funcao);
			novo->variaveis[pos] = v;
		}
	}

	free(*h);
	*h = novo;
}

void removeVarHash(varHash h) {

	if(!h)
		return;


	for (int i = 0, j = 0; i < h->n_elems; i++, j++)
	{
		while(h->variaveis[j] == NULL)
			j++;
		free(h->variaveis[j]);
	}
	free(h);
}

varInHash novaVar(char *nome, int tipo, int size, int endereco, char *funcao){

	varInHash var 	= (varInHash)malloc(sizeof(struct variable));

	var->nome 		= strdup(nome);
	var->tipo 		= tipo;
	var->size 		= size;
	var->endereco 	= endereco;
	var->funcao 	= strdup(funcao);

	return var;
}

varInHash copVar(varInHash v){

	varInHash var 	= (varInHash)malloc(sizeof(struct variable));

	var->nome 		= strdup(v->nome);
	var->tipo 		= v->tipo;
	var->size 		= v->size;
	var->endereco 	= v->endereco;
	var->funcao 	= strdup(v->funcao);

	return var;
}

int poeVar(varHash *h, varInHash v) {
	int ret = -1;
	int e = existeVar(h, v->nome, v->funcao);

	if ((*h)->n_elems >= (*h)->size / 2)
		resize_varHash(h);

	if (e != -1){
		(*h)->variaveis[e] = v;
		ret = 0;
	} else {
		int pos = getPos((*h)->size, v->nome);

		while ((*h)->variaveis[pos % (*h)->size] != NULL)
			pos++;

		(*h)->variaveis[pos] = v;
		(*h)->n_elems++;
	}
	return ret;
}

int adVar(varHash *h, char* nome, int tipo, int size, int endereco, char* funcao){
	int ret;
	varInHash v = novaVar(nome, tipo, size, endereco, funcao);
	ret = poeVar(h,v);

	return ret;
}

int existeVar(varHash *h, char* nome, char* funcao) {
	int key = getPos((*h)->size, nome);
	int res = -1;
	int i = 0;

	if ((*h)->variaveis[key] != NULL) {
		while( res == -1 && i < (*h)->size && (*h)->variaveis[key % (*h)->size] != NULL ){
			if(strcmp((*h)->variaveis[key]->nome,nome) == 0 && strcmp((*h)->variaveis[key]->funcao,funcao) == 0 )
				res = key;
			else{
				key++;
				i++;
			}
		}
	}

	return res;
}

varInHash tiraVar(varHash *h, char* nome, char *funcao){
	int pos = existeVar(h, nome, funcao);

	if (pos == -1)
		return NULL;
	else
		return (*h)->variaveis[pos];
}

void printHash(varHash *h) {

	int i = 0;
	int j = 0;

	printf("Hash:\n");
	printf("Tamanho Hash: %d\nNo Elementos: %d\n", (*h)->size, (*h)->n_elems);

	while (i < (*h)->size) {
		if ((*h)->variaveis[i] == NULL){
			printf("Elemento vazio\n");
		}
		else {
			printf("Variavel %s\n", (*h)->variaveis[i]->nome);
			printf("tipo: %d\n", (*h)->variaveis[i]->tipo);
			printf("Size: %d\n", (*h)->variaveis[i]->size);
			printf("Endereco: %d\n", (*h)->variaveis[i]->endereco);
			printf("Funcao: %s\n", (*h)->variaveis[i]->funcao);
		}
		i++;
	}
}


int getendereco(varHash *h, char* nome, char *funcao){
			varInHash v = (varInHash)malloc(sizeof(varInHash));
			v=tiraVar(h,nome,funcao);
			return v->endereco;
}

int getVarSize(varHash *h, char* nome, char* funcao) {
	varInHash v = (varInHash)malloc(sizeof(varInHash));
	v=tiraVar(h,nome,funcao);
	return v->size;
}



\end{lstlisting}
\newpage
\section{Auxiliar para Geração de Labels .h}
\begin{lstlisting}
#ifndef __STACK_H__
#define __STACK_H__

#include <stdlib.h>
#include <stdio.h>

typedef struct stack_node *PtStkNode;
typedef struct stack *PtStk;

PtStk stackCreate();
int stackDestroy(PtStk stack);
int stackPush(PtStk stack, int i);
int stackPop(PtStk stack);
int stackTop(PtStk stack);

#endif

\end{lstlisting}

\subsection{.c}
\begin{lstlisting}
#include "stack.h"

struct stack_node{
    int nr;
    struct stack_node * prev;
};

struct stack{
    struct stack_node * Top;
};


PtStk stackCreate(){

    PtStk stack;
    if((stack = (PtStk) malloc(sizeof(struct stack)))== NULL){
        fprintf(stderr, "Erro a alocar espaco na memoria para a stack\n");
        return NULL;
    }

    stack->Top = NULL;
    return stack;
}

int stackDestroy(PtStk stack){
    PtStk temp = stack;
    PtStkNode tempNode;

    if(temp == NULL){ return -1;}

    while(temp->Top != NULL){
        tempNode = temp->Top;
        temp->Top = temp->Top->prev;
        free(tempNode);
    }
    free(temp);
    stack = NULL;
    return 1;
}

int stackPush(PtStk stack, int i){
    PtStkNode tempNode;
    if(stack == NULL) return -1;
    if((tempNode = (PtStkNode) malloc(sizeof(struct stack_node)))==NULL){
        fprintf(stderr, "Erro a alocar memoria\n");
        return -2;
    }
    tempNode->nr = i;
    tempNode->prev = stack->Top;
    stack->Top = tempNode;

    return 1;
}

int stackPop(PtStk stack){
    PtStkNode tempNode;
    int res;
    if(stack == NULL){ return 0;}
    if(stack->Top == NULL){ return -1;}
    res = stack->Top->nr;
    tempNode = stack->Top;
    stack->Top = stack->Top->prev;
    free(tempNode);
    return res;
}

int stackTop(PtStk stack){
    if(stack==NULL){return -1;}
    return stack->Top->nr;    
}
\end{lstlisting}


\end{document} 
